\title{Distributed Algorithms CPSC-561 \\ Assignment 2}
\author{Michael McCulloch}
\date{}

\documentclass[10pt, letterpaper]{article}
\usepackage[letterpaper,margin=0.75in]{geometry}
\usepackage{listings}
\usepackage{algorithm}
\usepackage{algpseudocode}
\usepackage{amsfonts}
\usepackage{enumitem}
\lstset
{ %Formatting for code in appendix
    language=Java,
    basicstyle=\footnotesize,
    numbers=left,
    stepnumber=1,
    showstringspaces=false,
    tabsize=1,
    breaklines=true,
    breakatwhitespace=false,
}

\begin{document}
\maketitle

\section{LL/SC}


For all of the following algorithms, they are each wait-free, as they do not contain a loop.
\subsection*{A) CAS from SCAS}
\lstinputlisting[language=Java]{Code/1A.java}
\paragraph{} Let the invocation of SCAS.scas() be the linearization point X. \textit{Claim:} If a process P executes X before another process Q executes X, then the invocation of P is ordered before the invocation of Q. \textit{Proof:} By the definition of SCAS, if it returns the same value as it's first argument, it must have succeeded. If this value is different, another process Q must have finished it's invocation before the P's invocation of SCAS and after the assignment of it's result to \texttt{last}, a contradiction.

\subsection*{B) SCAS from CAS}
\lstinputlisting[language=Java]{Code/1B.java}
\paragraph{} Let the invocation of CAS.cas() be the linearization point X. \textit{Claim: see above}. \textit{Proof:} If CAS is successful, then the old value passed in must be the old value. If CAS is unsuccessful, then a different value must have been the old value, which can only be the case if another process executed X first. A contradiction.

\subsection*{C) CAS from LL/SC} 
\lstinputlisting[language=Java]{Code/1C.java}
\paragraph{} Let the invocation of \texttt{LLSC.ll()} be the linearization point X. \textit{Claim:} If a process P executes X after another process Q, then it is ordered after Q. \textit{Proof:} If any process P executes \texttt{LLSC.ll()} at $t$, then the only next successfull invocation of \texttt{LLSC.sc()} will be P's at $t'$, unless another process Q has executed \texttt{LLSC.ll()} between $t$ and $t'$, in which case Q's invocation of \texttt{LLSC.sc()} will be the successful one.


\section{$\mathbb{Z}_k$-Counter}
Solution for consensus no. 2:
\begin{algorithm}
\caption{Decision procedure 2 processes}\label{euclid}
\begin{algorithmic}[1]
\Procedure{Decide$_i$}{}
    \State $Val[i] \gets V_i$
    \State $x\gets Z_k.Inc() \bmod 2$
    \If{$x = 0$}
        \State \textbf{return} $Val[i]$
    \Else
        \State \textbf{return} $Val[1 - i]$
    \EndIf
\EndProcedure
\end{algorithmic}
\end{algorithm}

\paragraph{} \textit{Claim:} No consensus-3 solution exists using only $Z_k$ counters and registers. \textit{Proof by cases:} 
\begin{enumerate}
    \item 
\end{enumerate}
For three processes, A, B and C let X be a critical configureation and $\pi_I$ be the step that process $I$ takes to increment the counter. Then to a third process C, the branches $\pi_A\pi_B(X)$ and $\pi_B\pi_A(X)$ are indistinguishable, which contradicts X being critical.

\section{SRSW}
\subsection{A} 
\begin{enumerate}[topsep=0pt,itemsep=-1ex,partopsep=1ex,parsep=1ex]
    \item $w$ : W.write(5)
    \item $w$ : choose c
    \item[\texttt{IF}] c $\geq$ 5 \texttt{THEN}
    \begin{enumerate}[topsep=0pt,itemsep=-1ex,partopsep=1ex,parsep=1ex]
        \item[3.] $r$ : R.read()
        \item[4.] $w$ : W.write(c)
    \end{enumerate}
    \item[\texttt{ELSE}]
    \begin{enumerate}[topsep=0pt,itemsep=-1ex,partopsep=1ex,parsep=1ex]
        \item[3.] $w$ : W.write(c)
        \item[4.] $r$ : R.read()
    \end{enumerate}
\end{enumerate}
There are two ways in which $r$ may read 5, but no ways it may read 6 \\
Expected Value = (1/6)*1 + (1/6)*2 + (1/6)*3 + (1/6)*4 + (2/6)*5 \\
Expected Value = (1/6) * 10 + (2/6) * 5 \\
Expected Value = 20/6 = 3.3 \\
\subsection{B}
\begin{enumerate}[topsep=0pt,itemsep=-1ex,partopsep=1ex,parsep=1ex]
    \item $r$ : R.read()$_{5-9}$
    \item $w$ : W.write(5)
    \item $w$ : choose c
    \item[\texttt{IF}] c $\geq$ 4 
    \item[\texttt{THEN:}]  
    \begin{enumerate}[topsep=0pt,itemsep=-1ex,partopsep=1ex,parsep=1ex]
        \item[4.] $r$ : R.read()$_{10-13}$
        \item[5.] $w$ : W.write(c)
    \end{enumerate}
    \item[\texttt{ELSE}:]
    \begin{enumerate}[topsep=0pt,itemsep=-1ex,partopsep=1ex,parsep=1ex]
        \item[4.] $w$ : W.write(c)
        \item[5.] $r$ : R.read()$_{10-13}$
    \end{enumerate}
\end{enumerate}
\paragraph{} There are now 3 ways in which $r$ may read 4, and no ways it may read 5 or 6: \\
Expected Value = (1/6) * (1+2+3) + (3/6) * 4 = 3

\end{document}