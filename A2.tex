\title{Distributed Algorithms CPSC-561 \\ Assignment 2}
\author{Michael McCulloch}
\date{}

\documentclass[11pt]{article}
\usepackage{listings}
\usepackage{algorithm}
\usepackage{algpseudocode}
\usepackage{amsfonts}
\begin{document}
\maketitle

\section{LL/SC}
For all of the following algorithms, they are each wait-free, as they do not contain a loop.
\subsection*{A) CAS from SCAS}
\begin{algorithm}
    \caption{CAS from SCAS}
    \begin{algorithmic}[1]
    \Function{$CAS_p$}{old,new}
        \State old' = SCAS.scas(old, new)
        \State \Return old' = old
    \EndFunction
    \end{algorithmic}
\end{algorithm}
\paragraph{} Let the invocation of SCAS.scas() be the linearization point X. \textit{Claim:} If a process P executes X before another process Q executes X, then the invocation of P is ordered before the invocation of Q. \textit{Proof:} By the definition of SCAS, if it returns the same value as it's first argument, it must have succeeded. If this value is different, another process Q must have finished it's invocation between the invocation of SCAS and the assignment of it's result to old' within P, a contradiction.

\pagebreak
\subsection*{B) SCAS from CAS}
\begin{algorithm}
    \caption{SCAS from CAS}
    \begin{algorithmic}[1]
    \Function{$SCAS_p$}{old,new}
        \State success = CAS.cas(old, new)
        \If{success}
            \State \Return old
        \Else 
            \State \Return CAS.read()
        \EndIf
    \EndFunction
    \end{algorithmic}
\end{algorithm}
\paragraph{} Let the invocation of CAS.cas() be the linearization point X. \textit{Claim: see above}. \textit{Proof:} If CAS is successful, then the old value passed in must be the old value. If CAS is unsuccessful, then a different value must have been the old value, which can only have been cause if another process executed X first. A contradiction.

\subsection*{C) CAS from LL/SC} 
\begin{algorithm}
    \caption{CAS from LL/SC}
    \begin{algorithmic}[1]
        \Function{$CAS_p$}{old, new}
            \State old' = LL/SC.LL()
            \If{old' $\neq$ old}
                \State \Return false
            \Else
                \State \Return LL/SC.SC(new)
            \EndIf
        \EndFunction
    \end{algorithmic}
\end{algorithm}
\paragraph{} \textit{Claim:} If a process P executes lines two through six, before another process Q executes the same, then P's invocation is ordered before Q's. \textit{Proof:} If P returns \texttt{false}, then another process must have executed either LL() or SC() successfully ...FILL IN THE BLANK...?

\section{$\mathbb{Z}_k$-Counter}
\paragraph{Claim:} There is no consensus-3 solution using $\mathbb{Z}_k$-counters.
Proof by cases for 3 processes: A, B and C.
\subsection{Case: Different $\mathbb{Z}_k$-counters}
    w.l.o.g. assume A and B increment different $\mathbb{Z}_k$-counters. Let $K$ be a bivalent configuration, $\pi_A$ and $\pi_b$ be the steps that A and B take, respectively, to increment their own counters ...FILL IN THE BLANK... then $\pi_A(K)$ and $\pi_A\pi_B(K)$ are indistinguishable to A
\subsection{Case: Different/Same Registers} 
    \textit{See proof in lecture notes.}
\subsection{Case: Same $\mathbb{Z}_k$-counters}
    A and B increment A $\mathbb{Z}_k$-counters. Let $K$ be a bivalent configuration, $\pi_A$ and $\pi_b$ be the steps that A and B take, respectively, to increment the counter. Assume for contradiction that $\pi_R$ causes the configuration to be $R$-critical. If the counter's initial value is $c$, then $\pi_A(K)$ causes it to be $c+1$ mod $k$. the same is true for $\pi_B(K)$. Hence, after $\pi_A\pi_B(K)$, the value of the counter is $c+2$ mod $k$ and the same is true for $\pi_B\pi_A(K)$. Therefor, to a third process C, these two states are identical, and therefor are not critical, yielding a contradiction.

\section{SRSW}

\end{document}